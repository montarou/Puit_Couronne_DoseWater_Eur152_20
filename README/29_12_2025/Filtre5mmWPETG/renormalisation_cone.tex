\documentclass[a4paper,11pt]{article}
\usepackage[utf8]{inputenc}
\usepackage[T1]{fontenc}
\usepackage[french]{babel}
\usepackage{amsmath, amssymb}
\usepackage{tikz}
\usepackage{geometry}
\usepackage{xcolor}
\usepackage{booktabs}
\usepackage{siunitx}
\usepackage{fancybox}

\geometry{margin=2.5cm}

\usetikzlibrary{patterns, arrows.meta, calc, decorations.pathmorphing}

\definecolor{source}{RGB}{255,200,0}
\definecolor{cone20}{RGB}{100,150,255}
\definecolor{cone60}{RGB}{255,150,100}
\definecolor{water}{RGB}{100,180,255}

\title{Renormalisation pour une émission\\dans un cône restreint\\[0.5em]
\large Simulation Monte Carlo -- Source Eu-152}
\author{}
\date{}

\begin{document}
\maketitle

\section{Problématique}

\subsection{Contexte}

La source Eu-152 a une activité de $A = 44$ kBq sur $4\pi$ stéradians (émission isotrope). Cependant, pour optimiser le temps de calcul de la simulation Monte Carlo, on souhaite restreindre l'émission des gammas dans un cône de demi-angle $\theta_{\text{cone}} = 20°$ dirigé vers le détecteur (couronnes d'eau).

\subsection{Question}

Comment relier les résultats obtenus avec $N_{\text{sim}}$ événements simulés dans le cône de 20° à un temps d'irradiation réel correspondant à la source isotrope de 44 kBq ?

\section{Calcul de l'angle solide}

\subsection{Formule générale}

L'angle solide $\Omega$ d'un cône de demi-angle $\theta$ vu depuis son sommet est :

\begin{equation}
\boxed{\Omega = 2\pi \left(1 - \cos\theta\right)}
\end{equation}

\subsection{Application numérique}

\subsubsection{Cône de 20° (simulation optimisée)}

\begin{align}
\Omega_{20°} &= 2\pi \left(1 - \cos 20°\right) \\
&= 2\pi \left(1 - 0.9397\right) \\
&= 2\pi \times 0.0603 \\
&= \boxed{0.379 \text{ sr}}
\end{align}

\subsubsection{Sphère complète}

\begin{equation}
\Omega_{4\pi} = 4\pi = 12.566 \text{ sr}
\end{equation}

\subsection{Fraction de l'angle solide}

La fraction de l'émission $4\pi$ couverte par le cône de 20° est :

\begin{equation}
f = \frac{\Omega_{20°}}{\Omega_{4\pi}} = \frac{0.379}{12.566} = \boxed{0.0302 = 3.02\%}
\end{equation}

\begin{center}
\begin{tikzpicture}[scale=1.2, >=Stealth]
    % Source
    \fill[source] (0,0) circle (0.15);
    \node[below] at (0, -0.3) {Source};
    
    % Cône 20°
    \fill[cone20, opacity=0.4] (0,0) -- (5, {5*tan(20)}) arc (20:-20:5) -- cycle;
    \draw[thick, cone20] (0,0) -- (5, {5*tan(20)});
    \draw[thick, cone20] (0,0) -- (5, {-5*tan(20)});
    
    % Arc 20°
    \draw[thick, cone20] (1.5, 0) arc (0:20:1.5);
    \draw[thick, cone20] (1.5, 0) arc (0:-20:1.5);
    \node[cone20, right] at (1.7, 0.4) {\footnotesize $20°$};
    
    % Sphère 4π (représentation partielle)
    \draw[dashed, gray] (0,0) circle (3);
    \node[gray] at (2.5, 2.5) {$4\pi$};
    
    % Axe
    \draw[->, thick] (0,0) -- (5.5, 0) node[right] {$z$};
    
    % Eau (cible)
    \fill[water, opacity=0.6] (4.5, -1.2) rectangle (5, 1.2);
    \draw[thick, blue] (4.5, -1.2) rectangle (5, 1.2);
    \node[blue, right] at (5.1, 0) {\footnotesize Eau};
    
    % Annotations
    \node[draw, fill=white, anchor=north west] at (-1, -2) {
        \begin{tabular}{ll}
        $\Omega_{20°}$ & $= 0.379$ sr \\
        $f$ & $= 3.02\%$ de $4\pi$
        \end{tabular}
    };
\end{tikzpicture}
\end{center}

\section{Principe de la renormalisation}

\subsection{Équivalence physique}

Lorsqu'on simule $N_{\text{sim}}$ événements (désintégrations) dans un cône de demi-angle $\theta_{\text{cone}}$, on échantillonne uniquement la fraction $f$ de l'émission totale.

\textbf{Principe fondamental :} Ces $N_{\text{sim}}$ événements dans le cône correspondent au même nombre de gammas qu'une source isotrope aurait émis dans ce même cône après avoir effectué $N_{4\pi}$ désintégrations sur $4\pi$.

\begin{equation}
\boxed{N_{4\pi} = \frac{N_{\text{sim}}}{f}}
\end{equation}

\subsection{Interprétation}

\begin{itemize}
    \item $N_{\text{sim}}$ = nombre d'événements simulés dans le cône
    \item $N_{4\pi}$ = nombre équivalent de désintégrations de la source isotrope
    \item $f$ = fraction de l'angle solide ($\Omega_{\text{cone}} / 4\pi$)
\end{itemize}

\textbf{Exemple :} Si on simule $N_{\text{sim}} = 100\,000$ événements dans un cône de 20° :
\begin{equation}
N_{4\pi} = \frac{100\,000}{0.0302} = 3.31 \times 10^6 \text{ désintégrations sur } 4\pi
\end{equation}

\section{Calcul du temps d'irradiation}

\subsection{Relation activité -- temps}

L'activité $A$ de la source est définie comme le nombre de désintégrations par seconde :

\begin{equation}
A = \frac{N_{\text{désintégrations}}}{\Delta t}
\end{equation}

Donc le temps correspondant à $N_{4\pi}$ désintégrations est :

\begin{equation}
\boxed{T_{\text{irr}} = \frac{N_{4\pi}}{A} = \frac{N_{\text{sim}}}{f \cdot A}}
\end{equation}

\subsection{Prise en compte du nombre de gammas par désintégration}

Pour l'Eu-152, chaque désintégration produit en moyenne $\bar{n}_\gamma = 1.924$ gammas (dans le spectre considéré). Si la simulation génère des \textit{gammas} (et non des désintégrations), il faut en tenir compte :

\begin{equation}
\boxed{T_{\text{irr}} = \frac{N_{\text{sim}}}{f \cdot A \cdot \bar{n}_\gamma}}
\end{equation}

\subsection{Application numérique}

\begin{center}
\begin{tabular}{ll}
\toprule
\textbf{Paramètre} & \textbf{Valeur} \\
\midrule
Activité $A$ & $44\,000$ Bq \\
Demi-angle du cône $\theta$ & $20°$ \\
Fraction $f$ & $0.0302$ \\
Gammas par désintégration $\bar{n}_\gamma$ & $1.924$ \\
\bottomrule
\end{tabular}
\end{center}

\subsubsection{Temps par événement simulé}

\begin{align}
t_1 &= \frac{1}{f \cdot A \cdot \bar{n}_\gamma} \\
&= \frac{1}{0.0302 \times 44\,000 \times 1.924} \\
&= \frac{1}{2556} \text{ s} \\
&= \boxed{0.391 \text{ ms par événement}}
\end{align}

\subsubsection{Temps pour $N_{\text{sim}}$ événements}

\begin{equation}
\boxed{T_{\text{irr}} = N_{\text{sim}} \times 0.391 \text{ ms}}
\end{equation}

\begin{center}
\begin{tabular}{rr}
\toprule
$N_{\text{sim}}$ & $T_{\text{irr}}$ \\
\midrule
$10\,000$ & $3.91$ s \\
$100\,000$ & $39.1$ s \\
$1\,000\,000$ & $6.52$ min \\
$10\,000\,000$ & $1.09$ h \\
\bottomrule
\end{tabular}
\end{center}

\section{Formules récapitulatives}

\subsection{Formule générale du temps d'irradiation}

\begin{equation}
\boxed{T_{\text{irr}} = \frac{N_{\text{sim}} \cdot 4\pi}{2\pi(1-\cos\theta_{\text{cone}}) \cdot A \cdot \bar{n}_\gamma} = \frac{2 \cdot N_{\text{sim}}}{(1-\cos\theta_{\text{cone}}) \cdot A \cdot \bar{n}_\gamma}}
\end{equation}

\subsection{Formule du débit de dose}

Si la simulation donne une dose totale $D_{\text{sim}}$ (en Gy) pour $N_{\text{sim}}$ événements, le débit de dose est :

\begin{equation}
\boxed{\dot{D} = \frac{D_{\text{sim}}}{T_{\text{irr}}} = \frac{D_{\text{sim}} \cdot f \cdot A \cdot \bar{n}_\gamma}{N_{\text{sim}}}}
\end{equation}

\subsection{Vérification dimensionnelle}

\begin{align}
[\dot{D}] &= \frac{[\text{Gy}] \times [\text{sr}] \times [\text{Bq}] \times [1]}{[\text{sr}] \times [1]} \\
&= \frac{\text{Gy} \times \text{s}^{-1}}{1} = \text{Gy/s}
\end{align}

\section{Comparaison des configurations}

\begin{center}
\begin{tabular}{lccc}
\toprule
\textbf{Configuration} & $\theta$ & $f$ & $T_{\text{irr}}$ pour $10^5$ évts \\
\midrule
Cône actuel (60°) & $60°$ & $25.0\%$ & $4.72$ s \\
Cône optimisé (20°) & $20°$ & $3.02\%$ & $39.1$ s \\
Cône très restreint (10°) & $10°$ & $0.76\%$ & $155$ s \\
Isotrope ($4\pi$) & $180°$ & $100\%$ & $1.18$ s \\
\bottomrule
\end{tabular}
\end{center}

\textbf{Interprétation :} Plus le cône est restreint, plus chaque événement simulé représente un temps d'irradiation long (car on concentre les gammas simulés dans une direction utile).

\section{Implémentation dans le code Geant4}

\subsection{Modification de PrimaryGeneratorAction}

Pour changer le demi-angle du cône d'émission, modifier dans \texttt{PrimaryGeneratorAction.cc} :

\begin{verbatim}
// Ancien (60°)
G4double maxCosTheta = std::cos(60.0 * deg);

// Nouveau (20°)
G4double maxCosTheta = std::cos(20.0 * deg);
\end{verbatim}

\subsection{Calcul automatique dans RunAction}

Modifier \texttt{RunAction.cc} pour calculer le temps d'irradiation correct :

\begin{verbatim}
// Paramètres
G4double coneAngle = 20.0 * deg;           // Demi-angle du cône
G4double activity = 44000.0;               // Bq
G4double gammasPerDecay = 1.924;

// Fraction de l'angle solide
G4double f = (1.0 - std::cos(coneAngle)) / 2.0;

// Temps d'irradiation
G4double T_irr = nofEvents / (f * activity * gammasPerDecay);
\end{verbatim}

\section{Résumé et formules clés}

\begin{center}
\shadowbox{
\begin{minipage}{0.9\textwidth}
\textbf{Formules essentielles pour la renormalisation}

\vspace{0.5em}

\textbf{1. Fraction de l'angle solide :}
\begin{equation*}
f = \frac{1 - \cos\theta_{\text{cone}}}{2}
\end{equation*}

\textbf{2. Temps d'irradiation équivalent :}
\begin{equation*}
T_{\text{irr}} = \frac{N_{\text{sim}}}{f \cdot A \cdot \bar{n}_\gamma}
\end{equation*}

\textbf{3. Application pour $\theta = 20°$, $A = 44$ kBq, $\bar{n}_\gamma = 1.924$ :}
\begin{equation*}
T_{\text{irr}} = N_{\text{sim}} \times 0.391 \text{ ms}
\end{equation*}

\textbf{4. Débit de dose :}
\begin{equation*}
\dot{D} = \frac{D_{\text{sim}}}{T_{\text{irr}}}
\end{equation*}
\end{minipage}
}
\end{center}

\newpage
\section{Annexe : Schéma géométrique}

\begin{figure}[htbp]
\centering
\begin{tikzpicture}[scale=0.7, >=Stealth]

    % Fond
    \fill[gray!5] (-2,-5) rectangle (14,5);
    
    % Axe z
    \draw[->, ultra thick] (-1,0) -- (13,0) node[right] {\textbf{z}};
    
    % Source
    \fill[source] (2, 0) circle (0.2);
    \node[below=0.3cm] at (2, 0) {Source};
    \node[above=0.2cm] at (2, 0) {Eu-152};
    
    % Cône 60° (ancien, en transparence)
    \fill[cone60, opacity=0.15] (2,0) -- (12, {10*tan(60)}) -- (12, {-10*tan(60)}) -- cycle;
    \draw[thick, cone60, dashed] (2,0) -- (8, {6*tan(60)});
    \draw[thick, cone60, dashed] (2,0) -- (8, {-6*tan(60)});
    \node[cone60] at (5, 3.5) {\footnotesize Ancien cône 60°};
    
    % Cône 20° (nouveau)
    \fill[cone20, opacity=0.4] (2,0) -- (12, {10*tan(20)}) -- (12, {-10*tan(20)}) -- cycle;
    \draw[ultra thick, cone20] (2,0) -- (12, {10*tan(20)});
    \draw[ultra thick, cone20] (2,0) -- (12, {-10*tan(20)});
    
    % Arc 20°
    \draw[ultra thick, cone20] ({2+2}, 0) arc (0:20:2);
    \draw[ultra thick, cone20] ({2+2}, 0) arc (0:-20:2);
    \node[cone20, right] at (4.3, 0.5) {\footnotesize $20°$};
    
    % Filtre
    \fill[gray!50] (3.75, -2.5) rectangle (4.25, 2.5);
    \draw[thick] (3.75, -2.5) rectangle (4.25, 2.5);
    \node[above] at (4, 2.6) {\footnotesize Filtre};
    
    % Container + Eau
    \fill[gray!50] (9.65, 2.5) rectangle (10.55, 2.7);
    \fill[gray!50] (9.65, -2.5) rectangle (10.55, -2.7);
    \fill[gray!50] (10.35, -2.7) rectangle (10.55, 2.7);
    \fill[water] (9.85, -2.5) rectangle (10.35, 2.5);
    \draw[thick] (9.65, 2.5) rectangle (10.55, 2.7);
    \draw[thick] (9.65, -2.5) rectangle (10.55, -2.7);
    \draw[thick, blue] (9.85, -2.5) rectangle (10.35, 2.5);
    \node[above] at (10, 2.8) {\footnotesize Container + Eau};
    
    % Légende
    \node[draw, fill=white, anchor=north west] at (10.5, -2) {
        \footnotesize
        \begin{tabular}{ll}
        \textcolor{cone60}{---} & Cône 60° ($f=25\%$) \\
        \textcolor{cone20}{\rule{0.5cm}{2pt}} & Cône 20° ($f=3\%$) \\
        \end{tabular}
    };
    
    % Distances
    \draw[<->, thick] (2, -4) -- (9.85, -4);
    \node[below] at (5.9, -4) {\footnotesize 7.85 cm};

\end{tikzpicture}
\caption{Comparaison entre le cône d'émission de 60° (actuel) et le cône optimisé de 20°. Le cône de 20° concentre les gammas simulés vers le détecteur, améliorant l'efficacité statistique au prix d'un temps d'irradiation équivalent plus long par événement.}
\end{figure}

\section{Annexe : Tableau de conversion rapide}

Pour $\theta = 20°$, $A = 44$ kBq, $\bar{n}_\gamma = 1.924$ :

\begin{center}
\begin{tabular}{rrrl}
\toprule
$N_{\text{sim}}$ & $T_{\text{irr}}$ & & Commentaire \\
\midrule
$1\,000$ & $0.391$ s & & Test rapide \\
$10\,000$ & $3.91$ s & & Validation \\
$100\,000$ & $39.1$ s & $\approx 0.65$ min & Production légère \\
$500\,000$ & $196$ s & $\approx 3.3$ min & Production moyenne \\
$1\,000\,000$ & $391$ s & $\approx 6.5$ min & Production standard \\
$5\,000\,000$ & $1956$ s & $\approx 33$ min & Haute statistique \\
$10\,000\,000$ & $3912$ s & $\approx 1.1$ h & Très haute statistique \\
\bottomrule
\end{tabular}
\end{center}

\end{document}
