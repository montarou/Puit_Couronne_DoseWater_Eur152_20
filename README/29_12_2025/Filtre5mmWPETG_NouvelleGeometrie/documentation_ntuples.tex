\documentclass[11pt,a4paper]{article}
\usepackage[utf8]{inputenc}
\usepackage[T1]{fontenc}
\usepackage[french]{babel}
\usepackage{geometry}
\usepackage{booktabs}
\usepackage{array}
\usepackage{xcolor}
\usepackage{float}
\usepackage{tcolorbox}
\usepackage{caption}

\geometry{margin=2.5cm}

\begin{document}

\section*{Ntuples des plans de comptage du container}

\begin{tcolorbox}[colback=blue!5,colframe=blue,title=\textbf{Addition de ntuples}]
\noindent Deux ntuples ont été ajoutés pour enregistrer les passages de particules aux plans de comptage situés juste avant et juste après l'eau (GAP = 0). Les données sont enregistrées \textbf{par désintégration} (un événement = une désintégration).
\end{tcolorbox}

%===============================================================================
% NTUPLE PRECONTAINER
%===============================================================================

\subsection*{Ntuple \texttt{precontainer} (index 2)}

\noindent Ce ntuple enregistre les \textbf{photons} et les \textbf{électrons} traversant le plan PreContainerPlane en direction de l'eau.

\begin{tcolorbox}[colback=orange!10,colframe=orange!80!black,title=\textbf{Caractéristiques du plan PreContainerPlane}]
\begin{itemize}
    \item \textbf{Position} : z = 97.5 -- 98.5 mm (centre à 98.0 mm)
    \item \textbf{Limite haute} : 98.5 mm = surface basse de l'eau
    \item \textbf{Matériau} : Air
    \item \textbf{GAP} : 0 (collé à l'eau)
    \item \textbf{Rayon} : 25 mm
    \item \textbf{Volume logique} : \texttt{PreContainerPlaneLog}
\end{itemize}
\end{tcolorbox}

\begin{table}[H]
\centering
\caption{Variables du ntuple \texttt{precontainer}}
\begin{tabular}{llp{8cm}}
\toprule
\textbf{Variable} & \textbf{Type} & \textbf{Description} \\
\midrule
\texttt{nPhotons} & Int & Nombre de photons traversant le plan en direction +z (vers l'eau) \\
\texttt{sumEPhotons\_keV} & Double & Somme des énergies cinétiques de ces photons (keV) \\
\midrule
\texttt{nElectrons} & Int & Nombre d'électrons traversant le plan en direction +z (vers l'eau) \\
\texttt{sumEElectrons\_keV} & Double & Somme des énergies cinétiques de ces électrons (keV) \\
\bottomrule
\end{tabular}
\end{table}

\subsubsection*{Conditions de remplissage}

\noindent Pour chaque particule traversant le plan PreContainerPlane, les conditions suivantes sont vérifiées :

\paragraph{Photons vers l'eau}
\begin{itemize}
\item \textbf{Type de particule} : gamma ($\gamma$)
\item \textbf{Direction} : $p_z > 0$ (vers l'eau, direction +z)
\item \textbf{Condition de passage} : 
\begin{itemize}
\item Volume de départ $\neq$ \texttt{PreContainerPlaneLog}
\item Volume d'arrivée $=$ \texttt{PreContainerPlaneLog}
\end{itemize}
\end{itemize}

\begin{verbatim}
if (postLogVolName == "PreContainerPlaneLog" 
    && logicalVolumeName != "PreContainerPlaneLog") {
    if (particleName == "gamma" && pz > 0) {
        fEventAction->RecordPreContainerPhoton(kineticEnergy);
    }
}
\end{verbatim}

\paragraph{Électrons vers l'eau}
\begin{itemize}
\item \textbf{Type de particule} : électron ($e^-$)
\item \textbf{Direction} : $p_z > 0$ (vers l'eau, direction +z)
\item \textbf{Condition de passage} : 
\begin{itemize}
\item Volume de départ $\neq$ \texttt{PreContainerPlaneLog}
\item Volume d'arrivée $=$ \texttt{PreContainerPlaneLog}
\end{itemize}
\end{itemize}

\begin{verbatim}
if (postLogVolName == "PreContainerPlaneLog" 
    && logicalVolumeName != "PreContainerPlaneLog") {
    if (particleName == "e-" && pz > 0) {
        fEventAction->RecordPreContainerElectron(kineticEnergy);
    }
}
\end{verbatim}

\newpage
%===============================================================================
% NTUPLE POSTCONTAINER
%===============================================================================

\subsection*{Ntuple \texttt{postcontainer} (index 3)}

\noindent Ce ntuple enregistre les particules traversant le plan PostContainerPlane, avec distinction selon leur type et direction.

\begin{tcolorbox}[colback=violet!10,colframe=violet!80!black,title=\textbf{Caractéristiques du plan PostContainerPlane}]
\begin{itemize}
    \item \textbf{Position} : z = 103.5 -- 104.5 mm (centre à 104.0 mm)
    \item \textbf{Limite basse} : 103.5 mm = surface haute de l'eau
    \item \textbf{Matériau} : W/PETG (75\%/25\%)
    \item \textbf{GAP} : 0 (collé à l'eau)
    \item \textbf{Rayon} : 25 mm
    \item \textbf{Volume logique} : \texttt{PostContainerPlaneLog}
\end{itemize}
\end{tcolorbox}

\begin{table}[H]
\centering
\caption{Variables du ntuple \texttt{postcontainer}}
\begin{tabular}{llp{7cm}}
\toprule
\textbf{Variable} & \textbf{Type} & \textbf{Description} \\
\midrule
\texttt{nPhotons\_back} & Int & Nombre de photons venant de l'eau (direction -z) \\
\texttt{sumEPhotons\_back\_keV} & Double & Somme des énergies cinétiques de ces photons (keV) \\
\midrule
\texttt{nElectrons\_back} & Int & Nombre d'électrons venant de l'eau (direction -z) \\
\texttt{sumEElectrons\_back\_keV} & Double & Somme des énergies cinétiques de ces électrons (keV) \\
\midrule
\texttt{nElectrons\_fwd} & Int & Nombre d'électrons allant vers l'eau (direction +z) \\
\texttt{sumEElectrons\_fwd\_keV} & Double & Somme des énergies cinétiques de ces électrons (keV) \\
\bottomrule
\end{tabular}
\end{table}

\subsubsection*{Conditions de remplissage}

\noindent Pour chaque particule traversant le plan PostContainerPlane :

\paragraph{Photons rétrodiffusés (depuis l'eau)}
\begin{itemize}
\item \textbf{Type de particule} : gamma ($\gamma$)
\item \textbf{Direction} : $p_z < 0$ (depuis l'eau, direction -z)
\end{itemize}

\begin{verbatim}
if (postLogVolName == "PostContainerPlaneLog" 
    && logicalVolumeName != "PostContainerPlaneLog") {
    if (particleName == "gamma" && pz < 0) {
        fEventAction->RecordPostContainerPhotonBackward(kineticEnergy);
    }
}
\end{verbatim}

\paragraph{Électrons rétrodiffusés (depuis l'eau)}
\begin{itemize}
\item \textbf{Type de particule} : électron ($e^-$)
\item \textbf{Direction} : $p_z < 0$ (depuis l'eau, direction -z)
\end{itemize}

\begin{verbatim}
if (postLogVolName == "PostContainerPlaneLog" 
    && logicalVolumeName != "PostContainerPlaneLog") {
    if (particleName == "e-" && pz < 0) {
        fEventAction->RecordPostContainerElectronBackward(kineticEnergy);
    }
}
\end{verbatim}

\paragraph{Électrons vers l'eau}
\begin{itemize}
\item \textbf{Type de particule} : électron ($e^-$)
\item \textbf{Direction} : $p_z > 0$ (vers l'eau, direction +z)
\end{itemize}

\begin{verbatim}
if (postLogVolName == "PostContainerPlaneLog" 
    && logicalVolumeName != "PostContainerPlaneLog") {
    if (particleName == "e-" && pz > 0) {
        fEventAction->RecordPostContainerElectronForward(kineticEnergy);
    }
}
\end{verbatim}

\vspace{0.5cm}
\begin{tcolorbox}[colback=gray!10,colframe=gray!80!black,title=\textbf{Remarques importantes}]
\begin{itemize}
    \item Le plan \textbf{PostContainerPlane} est en \textbf{W/PETG} (même matériau que le container). Les particules traversant ce plan peuvent donc interagir avec le matériau du plan lui-même.
    \item Le plan \textbf{PreContainerPlane} est en \textbf{air}, donc transparent aux particules.
    \item Les deux plans sont collés à l'eau (\textbf{GAP = 0}).
    \item Les données sont enregistrées \textbf{par événement} (une ligne par désintégration).
\end{itemize}
\end{tcolorbox}

%===============================================================================
% SCHÉMA RÉCAPITULATIF
%===============================================================================

\newpage
\subsection*{Schéma récapitulatif des flux de particules}

\begin{center}
\begin{tikzpicture}[scale=0.8, >=latex]
    % Axe z
    \draw[->, thick] (-1,0) -- (12,0) node[right] {$z$ (mm)};
    
    % PreContainerPlane
    \fill[orange!40] (1,-2) rectangle (2,2);
    \draw[orange!80!black, thick] (1,-2) rectangle (2,2);
    \node[above, font=\footnotesize] at (1.5,2.2) {PreContainer};
    \node[below, font=\tiny] at (1.5,-2.2) {98 mm};
    \node[below, font=\tiny, orange!80!black] at (1.5,-2.7) {(Air)};
    
    % Eau
    \fill[blue!30] (3,-2) rectangle (7,2);
    \draw[blue!70!black, thick] (3,-2) rectangle (7,2);
    \node[font=\footnotesize] at (5,0) {\textbf{EAU}};
    \node[below, font=\tiny] at (5,-2.2) {98.5 -- 103.5 mm};
    
    % PostContainerPlane
    \fill[violet!40] (8,-2) rectangle (9,2);
    \draw[violet!80!black, thick] (8,-2) rectangle (9,2);
    \node[above, font=\footnotesize] at (8.5,2.2) {PostContainer};
    \node[below, font=\tiny] at (8.5,-2.2) {104 mm};
    \node[below, font=\tiny, violet!80!black] at (8.5,-2.7) {(W/PETG)};
    
    % Flèches PreContainer
    \draw[->, red, very thick] (-0.5,1) -- (1,1);
    \node[above, font=\tiny, red] at (0.25,1) {$\gamma$ +z};
    \draw[->, green!60!black, very thick] (-0.5,0) -- (1,0);
    \node[above, font=\tiny, green!60!black] at (0.25,0) {$e^-$ +z};
    
    % Flèches PostContainer (backward)
    \draw[<-, red, very thick] (7,1) -- (8,1);
    \node[above, font=\tiny, red] at (7.5,1) {$\gamma$ -z};
    \draw[<-, green!60!black, very thick] (7,-0.5) -- (8,-0.5);
    \node[below, font=\tiny, green!60!black] at (7.5,-0.5) {$e^-$ -z};
    
    % Flèches PostContainer (forward)
    \draw[->, green!60!black, very thick] (9,0.5) -- (10.5,0.5);
    \node[above, font=\tiny, green!60!black] at (9.75,0.5) {$e^-$ +z};
    
    % Source
    \draw[->, black, very thick] (-2,0) -- (-1,0);
    \node[left, font=\footnotesize] at (-2,0) {Source};
    
\end{tikzpicture}
\end{center}

\vspace{0.5cm}

\begin{table}[H]
\centering
\caption{Récapitulatif des variables par plan}
\begin{tabular}{lll}
\toprule
\textbf{Plan} & \textbf{Particule} & \textbf{Variables} \\
\midrule
\multirow{2}{*}{PreContainerPlane} & $\gamma$ vers eau (+z) & \texttt{nPhotons}, \texttt{sumEPhotons\_keV} \\
                                    & $e^-$ vers eau (+z) & \texttt{nElectrons}, \texttt{sumEElectrons\_keV} \\
\midrule
\multirow{3}{*}{PostContainerPlane} & $\gamma$ depuis eau (-z) & \texttt{nPhotons\_back}, \texttt{sumEPhotons\_back\_keV} \\
                                     & $e^-$ depuis eau (-z) & \texttt{nElectrons\_back}, \texttt{sumEElectrons\_back\_keV} \\
                                     & $e^-$ vers eau (+z) & \texttt{nElectrons\_fwd}, \texttt{sumEElectrons\_fwd\_keV} \\
\bottomrule
\end{tabular}
\end{table}

\end{document}
