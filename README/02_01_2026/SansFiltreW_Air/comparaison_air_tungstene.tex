\documentclass[11pt,a4paper]{article}

% ═══════════════════════════════════════════════════════════════
% PACKAGES
% ═══════════════════════════════════════════════════════════════
\usepackage[utf8]{inputenc}
\usepackage[T1]{fontenc}
\usepackage[french, provide=*]{babel}
\usepackage{amsmath,amssymb}
\usepackage{bm}
\usepackage{graphicx}
\usepackage{booktabs}
\usepackage{multirow}
\usepackage{siunitx}
\usepackage{xcolor}
\usepackage{colortbl}
\usepackage{geometry}
\usepackage{fancyhdr}
\usepackage{caption}
\usepackage{hyperref}
\usepackage{float}

% ═══════════════════════════════════════════════════════════════
% CONFIGURATION
% ═══════════════════════════════════════════════════════════════
\geometry{margin=2.5cm}
\hypersetup{
    colorlinks=true,
    linkcolor=blue!70!black,
    urlcolor=blue!70!black
}

\sisetup{
    locale = FR,
    per-mode = symbol,
    separate-uncertainty = true
}

% En-tête et pied de page
\pagestyle{fancy}
\fancyhf{}
\fancyhead[L]{\small Simulation Geant4 -- Comparaison Air/Tungstène}
\fancyhead[R]{\small \today}
\fancyfoot[C]{\thepage}

% Couleurs personnalisées
\definecolor{aircolor}{RGB}{200, 220, 255}
\definecolor{tungcolor}{RGB}{255, 220, 200}
\definecolor{ratiocolor}{RGB}{220, 255, 220}

% ═══════════════════════════════════════════════════════════════
% DOCUMENT
% ═══════════════════════════════════════════════════════════════
\begin{document}

% ───────────────────────────────────────────────────────────────
% TITRE
% ───────────────────────────────────────────────────────────────
\begin{center}
    {\LARGE\bfseries Comparaison des Résultats de Simulation\\[0.3em]
    Feuille de 20~µm : Air vs Tungstène}\\[1.5em]
    {\large Simulation Geant4 -- Source ${}^{152}$Eu -- 25$\times 10^6$ événements}\\[0.5em]
    {\small \today}
\end{center}

\vspace{1em}

% ═══════════════════════════════════════════════════════════════
\section{Configuration des simulations}
% ═══════════════════════════════════════════════════════════════




\paragraph{Interprétation :}

\subsection{Photons transmis (direction $+z$)}

\begin{table}[H]
\centering
\caption{Statistiques des photons transmis au PostContainerPlane}
\begin{tabular}{l>{\columncolor{aircolor}}c>{\columncolor{tungcolor}}c>{\columncolor{ratiocolor}}c}
\toprule
\textbf{Observable} & \textbf{Air} & \textbf{Tungstène} & \textbf{Ratio W/Air} \\
\midrule
Nombre moyen $\bm{\langle N_\gamma \rangle}$ & \num{1.028} & \num{1.027} & $\sim 1.00$ \\
Écart-type $\bm{\sigma}$ & \num{0.943} & \num{0.943} & $\sim 1.00$ \\
Énergie moyenne $\bm{\langle \Sigma E_\gamma \rangle}$ & \SI{1069}{keV} & \SI{1069}{keV} & $\sim 1.00$ \\
\bottomrule
\end{tabular}
\end{table}

\paragraph{Interprétation :}
La transmission des photons est quasi-identique car l'épaisseur de 20~µm de tungstène est très faible par rapport au libre parcours moyen des photons gamma de haute énergie ($\lambda \gg 20$~µm pour $E_\gamma > 100$~keV).


% ═══════════════════════════════════════════════════════════════
\section{Tableau récapitulatif}
% ═══════════════════════════════════════════════════════════════

\begin{table}[H]
\centering
\caption{Synthèse de la comparaison Air vs Tungstène (feuille 20~µm)}
\begin{tabular}{llccc}
\toprule
\textbf{Catégorie} & \textbf{Observable} & \textbf{Air} & \textbf{Tungstène} & \textbf{Ratio} \\
\midrule
\multirow{3}{*}{\textbf{Photons backscatter}} 
    & Nombre moyen & \num{8.29e-5} & \num{8.11e-3} & $\times 98$ \\
    & Énergie moyenne & \SI{153}{keV} & \SI{80.7}{keV} & $\times 0.53$ \\
    & Événements & $\sim 2\,000$ & $\sim 189\,000$ & $\times 95$ \\
\midrule
\multirow{3}{*}{\textbf{Électrons backscatter}} 
    & Nombre moyen & \num{1.21e-5} & \num{1.66e-3} & $\times 137$ \\
    & Énergie moyenne & \SI{35}{keV} & \SI{156}{keV} & $\times 4.5$ \\
    & Événements & $\sim 300$ & $\sim 37\,000$ & $\times 123$ \\
\midrule
\multirow{2}{*}{\textbf{Photons transmis}} 
    & Nombre moyen & \num{1.028} & \num{1.027} & $\sim 1.0$ \\
    & Énergie moyenne & \SI{1069}{keV} & \SI{1069}{keV} & $\sim 1.0$ \\
\midrule
\multirow{2}{*}{\textbf{Électrons transmis}} 
    & Nombre moyen & \num{3.25e-4} & \num{4.13e-4} & $\times 1.27$ \\
    & Énergie moyenne & \SI{364}{keV} & \SI{290}{keV} & $\times 0.80$ \\
\bottomrule
\end{tabular}
\end{table}

% ═══════════════════════════════════════════════════════════════
\section{Conclusion}
% ═══════════════════════════════════════════════════════════════

La comparaison entre les deux configurations démontre clairement l'effet du tungstène :

\begin{enumerate}
    \item \textbf{Backscatter multiplié par $\sim 100$} : Le tungstène (Z=74) est un excellent rétrodiffuseur comparé à l'air (Z$_{\text{eff}} \approx 7.4$). Le rapport des sections efficaces d'interaction évolue approximativement comme $Z^2$ pour l'effet photoélectrique et $Z$ pour l'effet Compton.
    
    \item \textbf{Fluorescence X visible} : Le pic à 60-80~keV dans le spectre des photons backscatter correspond aux raies K$_\alpha$ et K$_\beta$ du tungstène, signature caractéristique de ce matériau.
    
    \item \textbf{Transmission quasi-inchangée} : L'épaisseur de 20~µm est insuffisante pour atténuer significativement les photons gamma. Le libre parcours moyen dans le tungstène pour des photons de 500~keV est de l'ordre de 4~mm, soit 200 fois l'épaisseur de la feuille.
    
    \item \textbf{Production d'électrons secondaires} : Le tungstène génère beaucoup plus d'électrons par effet photoélectrique et Compton, avec des énergies plus élevées que dans l'air.
\end{enumerate}

\vspace{1em}
\noindent Ces résultats valident le bon fonctionnement de la simulation et confirment que la feuille de tungstène de 20~µm produit un backscatter significatif de photons et d'électrons, tout en laissant passer la quasi-totalité du flux gamma incident.

\end{document}
