\documentclass[11pt,a4paper]{article}
\usepackage[utf8]{inputenc}
\usepackage[T1]{fontenc}
\usepackage{amsmath,amssymb}
\usepackage{booktabs}
\usepackage{array}
\usepackage{multirow}
\usepackage{longtable}
\usepackage{geometry}
\usepackage{xcolor}
\usepackage{colortbl}
\usepackage{fancyhdr}
\usepackage{listings}
\usepackage{tcolorbox}

\geometry{margin=2.5cm}
\pagestyle{fancy}
\fancyhf{}
\fancyhead[L]{Simulation Puits Couronne}
\fancyhead[R]{Histogrammes et Ntuples}
\fancyfoot[C]{\thepage}

\definecolor{headerblue}{RGB}{41,128,185}
\definecolor{lightgray}{RGB}{245,245,245}

\lstset{
    basicstyle=\ttfamily\small,
    backgroundcolor=\color{lightgray},
    frame=single,
    breaklines=true
}

\title{\textbf{Structure des Données ROOT}\\
\large Simulation Geant4 -- Puits Couronne\\
Fichier de sortie : \texttt{puits\_couronne\_output.root}}
\author{Documentation technique}
\date{\today}

\begin{document}

\maketitle
\tableofcontents
\newpage

%===============================================================================
\section{Vue d'ensemble}
%===============================================================================

Le fichier ROOT \texttt{puits\_couronne\_output.root} contient :
\begin{itemize}
    \item \textbf{9 histogrammes} (H0 à H8)
    \item \textbf{3 ntuples} (EventData, GammaData, RingDoseData)
\end{itemize}

Ces données sont enregistrées pour chaque run de simulation et permettent une analyse détaillée de :
\begin{itemize}
    \item La génération des gammas primaires (spectre Eu-152)
    \item La transmission à travers le filtre W/PETG
    \item La dose déposée dans les 5 anneaux d'eau
\end{itemize}

%===============================================================================
\section{Histogrammes}
%===============================================================================

\subsection{Liste des histogrammes}

\begin{table}[h!]
\centering
\renewcommand{\arraystretch}{1.3}
\begin{tabular}{|c|l|c|c|c|}
\hline
\rowcolor{headerblue}
\textcolor{white}{\textbf{ID}} & \textcolor{white}{\textbf{Nom}} & \textcolor{white}{\textbf{Bins}} & \textcolor{white}{\textbf{Min}} & \textcolor{white}{\textbf{Max}} \\
\hline
H0 & \texttt{nGammasPerEvent} & 15 & $-0.5$ & $14.5$ \\
\hline
H1 & \texttt{energySpectrum} & 1500 & 0 & 1500 keV \\
\hline
H2 & \texttt{totalEnergyPerEvent} & 500 & 0 & 5000 keV \\
\hline
H3 & \texttt{doseRing0} & 200 & 0 & 200 keV \\
\hline
H4 & \texttt{doseRing1} & 200 & 0 & 200 keV \\
\hline
H5 & \texttt{doseRing2} & 200 & 0 & 200 keV \\
\hline
H6 & \texttt{doseRing3} & 200 & 0 & 200 keV \\
\hline
H7 & \texttt{doseRing4} & 200 & 0 & 200 keV \\
\hline
H8 & \texttt{doseTotalWater} & 500 & 0 & 500 keV \\
\hline
\end{tabular}
\caption{Liste des histogrammes dans le fichier ROOT}
\end{table}

\subsection{Description détaillée}

\subsubsection{H0 : nGammasPerEvent}
\begin{tcolorbox}[colback=lightgray,colframe=headerblue,title=Nombre de gammas par événement]
\textbf{Description :} Distribution du nombre de gammas primaires générés par désintégration.\\
\textbf{Remplissage :} \texttt{RunAction::RecordEventStatistics()}\\
\textbf{Valeur attendue :} Moyenne $\bar{n}_\gamma \approx 1.924$ (spectre Eu-152)
\end{tcolorbox}

\subsubsection{H1 : energySpectrum}
\begin{tcolorbox}[colback=lightgray,colframe=headerblue,title=Spectre en énergie des gammas]
\textbf{Description :} Spectre des énergies de tous les gammas primaires générés.\\
\textbf{Remplissage :} \texttt{RunAction::RecordEventStatistics()}\\
\textbf{Raies principales :} 40, 122, 245, 344, 779, 964, 1112, 1408 keV
\end{tcolorbox}

\subsubsection{H2 : totalEnergyPerEvent}
\begin{tcolorbox}[colback=lightgray,colframe=headerblue,title=Énergie totale par événement]
\textbf{Description :} Somme des énergies de tous les gammas primaires par désintégration.\\
\textbf{Remplissage :} \texttt{RunAction::RecordEventStatistics()}
\end{tcolorbox}

\subsubsection{H3--H7 : doseRing0 à doseRing4}
\begin{tcolorbox}[colback=lightgray,colframe=headerblue,title=Dose par anneau d'eau]
\textbf{Description :} Distribution des dépôts d'énergie (en keV) dans chaque anneau d'eau, par désintégration.\\
\textbf{Remplissage :} \texttt{RunAction::AddRingEnergy()}\\[0.5em]
\begin{tabular}{|c|c|c|}
\hline
\textbf{Histo} & \textbf{Anneau} & \textbf{Rayon (mm)} \\
\hline
H3 & Ring 0 & $r = 0-5$ \\
H4 & Ring 1 & $r = 5-10$ \\
H5 & Ring 2 & $r = 10-15$ \\
H6 & Ring 3 & $r = 15-20$ \\
H7 & Ring 4 & $r = 20-25$ \\
\hline
\end{tabular}
\end{tcolorbox}

\subsubsection{H8 : doseTotalWater}
\begin{tcolorbox}[colback=lightgray,colframe=headerblue,title=Dose totale dans l'eau]
\textbf{Description :} Distribution de l'énergie totale déposée dans l'ensemble des anneaux d'eau, par désintégration.\\
\textbf{Remplissage :} \texttt{RunAction::RecordEventStatistics()}\\
\textbf{Condition :} Uniquement si $E_{dep} > 0$
\end{tcolorbox}

%===============================================================================
\section{Ntuples}
%===============================================================================

\subsection{Ntuple 0 : EventData}

\begin{tcolorbox}[colback=lightgray,colframe=headerblue,title=Données par événement (désintégration)]
\textbf{Description :} Une ligne par événement (désintégration simulée).\\
\textbf{Remplissage :} \texttt{EventAction::EndOfEventAction()}
\end{tcolorbox}

\begin{table}[h!]
\centering
\renewcommand{\arraystretch}{1.2}
\begin{tabular}{|c|l|c|p{7cm}|}
\hline
\rowcolor{headerblue}
\textcolor{white}{\textbf{Col}} & \textcolor{white}{\textbf{Nom}} & \textcolor{white}{\textbf{Type}} & \textcolor{white}{\textbf{Description}} \\
\hline
0 & \texttt{eventID} & Int & Numéro de l'événement \\
\hline
1 & \texttt{nPrimaries} & Int & Nombre de gammas primaires générés \\
\hline
2 & \texttt{totalEnergy} & Double & Énergie totale des primaires (keV) \\
\hline
3 & \texttt{nTransmitted} & Int & Nombre de gammas transmis à travers le filtre \\
\hline
4 & \texttt{nAbsorbed} & Int & Nombre de gammas absorbés par le filtre \\
\hline
5 & \texttt{nScattered} & Int & Nombre de gammas diffusés (Compton) \\
\hline
6 & \texttt{nSecondaries} & Int & Nombre de particules secondaires détectées \\
\hline
7 & \texttt{totalWaterDeposit} & Double & Énergie déposée dans l'eau (keV) \\
\hline
\end{tabular}
\caption{Structure du ntuple EventData}
\end{table}

\subsection{Ntuple 1 : GammaData}

\begin{tcolorbox}[colback=lightgray,colframe=headerblue,title=Données par gamma primaire]
\textbf{Description :} Une ligne par gamma primaire émis.\\
\textbf{Remplissage :} \texttt{EventAction::EndOfEventAction()}
\end{tcolorbox}

\begin{table}[h!]
\centering
\renewcommand{\arraystretch}{1.2}
\begin{tabular}{|c|l|c|p{6.5cm}|}
\hline
\rowcolor{headerblue}
\textcolor{white}{\textbf{Col}} & \textcolor{white}{\textbf{Nom}} & \textcolor{white}{\textbf{Type}} & \textcolor{white}{\textbf{Description}} \\
\hline
0 & \texttt{eventID} & Int & Numéro de l'événement parent \\
\hline
1 & \texttt{gammaIndex} & Int & Index du gamma dans l'événement (0, 1, 2, ...) \\
\hline
2 & \texttt{energyInitial} & Double & Énergie initiale (keV) \\
\hline
3 & \texttt{energyUpstream} & Double & Énergie au plan upstream (keV) \\
\hline
4 & \texttt{energyDownstream} & Double & Énergie au plan downstream (keV) \\
\hline
5 & \texttt{theta} & Double & Angle polaire d'émission (deg) \\
\hline
6 & \texttt{phi} & Double & Angle azimutal d'émission (deg) \\
\hline
7 & \texttt{detectedUpstream} & Int & Détecté au plan upstream (0/1) \\
\hline
8 & \texttt{detectedDownstream} & Int & Détecté au plan downstream (0/1) \\
\hline
9 & \texttt{transmitted} & Int & Transmis sans perte d'énergie (0/1) \\
\hline
\end{tabular}
\caption{Structure du ntuple GammaData}
\end{table}

\textbf{Critère de transmission :} Un gamma est considéré comme transmis si :
\begin{equation}
\left| E_{\text{upstream}} - E_{\text{downstream}} \right| < 1 \text{ keV}
\end{equation}

\subsection{Ntuple 2 : RingDoseData}

\begin{tcolorbox}[colback=lightgray,colframe=headerblue,title=Dose par anneau par désintégration]
\textbf{Description :} Une ligne par événement avec la dose déposée dans chaque anneau.\\
\textbf{Remplissage :} \texttt{EventAction::EndOfEventAction()}
\end{tcolorbox}

\begin{table}[h!]
\centering
\renewcommand{\arraystretch}{1.2}
\begin{tabular}{|c|l|c|p{6cm}|}
\hline
\rowcolor{headerblue}
\textcolor{white}{\textbf{Col}} & \textcolor{white}{\textbf{Nom}} & \textcolor{white}{\textbf{Type}} & \textcolor{white}{\textbf{Description}} \\
\hline
0 & \texttt{eventID} & Int & Numéro de l'événement \\
\hline
1 & \texttt{nPrimaries} & Int & Nombre de gammas primaires \\
\hline
2 & \texttt{doseRing0} & Double & Énergie déposée dans Ring 0 (keV) \\
\hline
3 & \texttt{doseRing1} & Double & Énergie déposée dans Ring 1 (keV) \\
\hline
4 & \texttt{doseRing2} & Double & Énergie déposée dans Ring 2 (keV) \\
\hline
5 & \texttt{doseRing3} & Double & Énergie déposée dans Ring 3 (keV) \\
\hline
6 & \texttt{doseRing4} & Double & Énergie déposée dans Ring 4 (keV) \\
\hline
7 & \texttt{doseTotal} & Double & Énergie totale déposée dans l'eau (keV) \\
\hline
\end{tabular}
\caption{Structure du ntuple RingDoseData}
\end{table}

%===============================================================================
\section{Flux de données}
%===============================================================================

\subsection{Diagramme de remplissage}

\begin{center}
\begin{tabular}{|l|l|l|}
\hline
\rowcolor{headerblue}
\textcolor{white}{\textbf{Classe}} & \textcolor{white}{\textbf{Méthode}} & \textcolor{white}{\textbf{Données remplies}} \\
\hline
\multirow{2}{*}{SteppingAction} & \texttt{UserSteppingAction()} & Détection dans les plans \\
& & Dépôts d'énergie $\rightarrow$ EventAction \\
\hline
\multirow{5}{*}{EventAction} & \texttt{BeginOfEventAction()} & Reset des compteurs \\
& & Enregistrement des primaires \\
\cline{2-3}
& \texttt{EndOfEventAction()} & Ntuple 0 (EventData) \\
& & Ntuple 1 (GammaData) \\
& & Ntuple 2 (RingDoseData) \\
\hline
\multirow{2}{*}{RunAction} & \texttt{RecordEventStatistics()} & H0, H1, H2, H8 \\
\cline{2-3}
& \texttt{AddRingEnergy()} & H3--H7 \\
\hline
\end{tabular}
\end{center}

\subsection{Séquence temporelle}

Pour chaque événement :
\begin{enumerate}
    \item \texttt{BeginOfEventAction} : initialisation, lecture des vertex primaires
    \item \texttt{UserSteppingAction} : tracking de chaque particule, détection, dépôts
    \item \texttt{EndOfEventAction} : calcul des statistiques, remplissage des ntuples
    \item \texttt{RecordEventStatistics} : mise à jour des compteurs globaux, histogrammes
\end{enumerate}

%===============================================================================
\section{Exemples d'analyse ROOT}
%===============================================================================

\subsection{Lecture des histogrammes}

\begin{lstlisting}[language=C++]
TFile* f = TFile::Open("puits_couronne_output.root");

// Spectre en energie
TH1D* hSpectrum = (TH1D*)f->Get("energySpectrum");
hSpectrum->Draw();

// Dose dans l'anneau central
TH1D* hRing0 = (TH1D*)f->Get("doseRing0");
hRing0->Draw();
\end{lstlisting}

\subsection{Analyse des ntuples}

\begin{lstlisting}[language=C++]
// Ntuple EventData
TTree* tEvent = (TTree*)f->Get("EventData");
tEvent->Draw("totalWaterDeposit", "totalWaterDeposit>0");

// Ntuple GammaData - transmission en fonction de l'energie
TTree* tGamma = (TTree*)f->Get("GammaData");
tGamma->Draw("transmitted:energyInitial", "", "colz");

// Ntuple RingDoseData - correlation entre anneaux
TTree* tRing = (TTree*)f->Get("RingDoseData");
tRing->Draw("doseRing0:doseRing4", "doseRing0>0 && doseRing4>0");
\end{lstlisting}

\subsection{Calcul de la dose moyenne}

\begin{lstlisting}[language=C++]
// Dose moyenne dans l'anneau 2
TTree* tRing = (TTree*)f->Get("RingDoseData");
double meanDose = tRing->GetEntries("doseRing2>0") > 0 ?
    tRing->GetMean("doseRing2") : 0;
cout << "Dose moyenne Ring 2: " << meanDose << " keV" << endl;
\end{lstlisting}

%===============================================================================
\section{Compteurs de run (output.log)}
%===============================================================================

En plus du fichier ROOT, les compteurs suivants sont affichés dans \texttt{output.log} :

\begin{table}[h!]
\centering
\renewcommand{\arraystretch}{1.2}
\begin{tabular}{|l|p{8cm}|}
\hline
\rowcolor{headerblue}
\textcolor{white}{\textbf{Compteur}} & \textcolor{white}{\textbf{Description}} \\
\hline
\texttt{fRingTotalEnergy[i]} & Énergie totale déposée dans l'anneau $i$ (MeV) \\
\hline
\texttt{fRingEventCount[i]} & Nombre d'événements avec dépôt dans l'anneau $i$ \\
\hline
\texttt{fGammasPreFilterPlane} & Gammas traversant le plan pré-filtre \\
\hline
\texttt{fGammasPostFilterPlane} & Gammas traversant le plan post-filtre \\
\hline
\texttt{fGammasPreWaterPlane} & Gammas traversant le plan pré-eau \\
\hline
\texttt{fGammasPostWaterPlane} & Gammas traversant le plan post-eau \\
\hline
\end{tabular}
\caption{Compteurs de vérification par run}
\end{table}

\end{document}
