\documentclass[11pt,a4paper]{article}

% ═══════════════════════════════════════════════════════════════════════════════
% PACKAGES
% ═══════════════════════════════════════════════════════════════════════════════
\usepackage[utf8]{inputenc}
\usepackage[T1]{fontenc}
\usepackage[french]{babel}
\usepackage{amsmath,amssymb}
\usepackage{booktabs}
\usepackage{siunitx}
\usepackage{geometry}
\usepackage{xcolor}
\usepackage{fancyhdr}
\usepackage{float}

\geometry{margin=2.5cm}
\sisetup{locale=FR, separate-uncertainty=true}

% En-têtes
\pagestyle{fancy}
\fancyhf{}
\rhead{Source Eu-152 -- Calcul dosimétrique}
\lhead{Temps d'irradiation}
\rfoot{Page \thepage}

% ═══════════════════════════════════════════════════════════════════════════════
% DOCUMENT
% ═══════════════════════════════════════════════════════════════════════════════

\begin{document}

\title{\textbf{Estimation des temps d'irradiation\\pour une source d'Europium-152}}
\author{Dispositif Puits Couronne}
\date{\today}
\maketitle

% ═══════════════════════════════════════════════════════════════════════════════
\section{Paramètres de la simulation}
% ═══════════════════════════════════════════════════════════════════════════════

Les paramètres dosimétriques ont été déterminés par simulation Monte Carlo Geant4 avec 25 millions de désintégrations d'Europium-152.

\begin{table}[H]
\centering
\caption{Paramètres dosimétriques issus de la simulation}
\begin{tabular}{@{}lr@{}}
\toprule
\textbf{Paramètre} & \textbf{Valeur} \\
\midrule
Dose moyenne par désintégration (anneau central) & \SI{7e-2}{nGy/désintégration} \\
Activité de la source (novembre 2025) & \SI{42.4}{kBq} \\
Demi-vie de l'Europium-152 & 13.517 ans \\
\bottomrule
\end{tabular}
\end{table}

% ═══════════════════════════════════════════════════════════════════════════════
\section{Méthodologie de calcul}
% ═══════════════════════════════════════════════════════════════════════════════

\subsection{Conversion d'unités}

La conversion entre nanogray (nGy) et centigray (cGy) s'effectue selon :
\begin{equation}
1~\text{nGy} = 10^{-9}~\text{Gy} = 10^{-7}~\text{cGy}
\end{equation}

Ainsi, pour une dose cible $D_{\text{cible}}$ exprimée en cGy :
\begin{equation}
D_{\text{cible}}~(\text{nGy}) = D_{\text{cible}}~(\text{cGy}) \times 10^{7}
\end{equation}

\subsection{Nombre de désintégrations nécessaires}

Le nombre de désintégrations $N$ requis pour atteindre une dose cible est :
\begin{equation}
N = \frac{D_{\text{cible}}}{\dot{D}} = \frac{D_{\text{cible}}~(\text{cGy}) \times 10^{7}}{\SI{7e-2}{nGy}}
\end{equation}

\textbf{Application numérique pour 50 cGy :}
\begin{equation}
N = \frac{50 \times 10^{7}}{7 \times 10^{-2}} = \frac{5 \times 10^{8}}{7 \times 10^{-2}} = 7.14 \times 10^{9}~\text{désintégrations}
\end{equation}

\subsection{Temps d'irradiation}

Le temps d'irradiation $t$ est donné par :
\begin{equation}
t = \frac{N}{A}
\end{equation}

où $A$ est l'activité de la source en Becquerel (Bq = désintégrations/seconde).

\subsection{Correction de décroissance radioactive}

L'activité décroît selon :
\begin{equation}
A(t) = A_0 \times e^{-\frac{\ln 2}{T_{1/2}} \times t}
\end{equation}

Pour l'Europium-152 ($T_{1/2} = 13.517$ ans), la décroissance sur 2 mois est négligeable ($<1\%$). On utilise donc $A \approx \SI{42.4}{kBq} = \SI{4.24e4}{Bq}$.

% ═══════════════════════════════════════════════════════════════════════════════
\section{Résultats}
% ═══════════════════════════════════════════════════════════════════════════════

\subsection{Calcul pour 50 cGy}

\begin{equation}
t_{50~\text{cGy}} = \frac{7.14 \times 10^{9}}{4.24 \times 10^{4}} = 1.68 \times 10^{5}~\text{s} \approx \boxed{47~\text{heures} \approx 2~\text{jours}}
\end{equation}

\subsection{Calcul pour 5 cGy}

\begin{equation}
t_{5~\text{cGy}} = \frac{t_{50~\text{cGy}}}{10} = \frac{47}{10} = \boxed{4.7~\text{heures} \approx 4\text{h}~42\text{min}}
\end{equation}

\subsection{Tableau récapitulatif}

\begin{table}[H]
\centering
\caption{Temps d'irradiation en fonction de la dose cible (A = \SI{42.4}{kBq})}
\label{tab:temps_irradiation}
\begin{tabular}{@{}cccc@{}}
\toprule
\textbf{Dose cible} & \textbf{Dose cible} & \textbf{Désintégrations} & \textbf{Temps d'irradiation} \\
(cGy) & (nGy) & nécessaires & \\
\midrule
0.5  & $5 \times 10^{6}$  & $7.1 \times 10^{8}$  & 28 min \\
1    & $1 \times 10^{7}$  & $1.4 \times 10^{9}$  & 56 min \\
5    & $5 \times 10^{7}$  & $7.1 \times 10^{9}$  & \textbf{4 h 42 min} \\
10   & $1 \times 10^{8}$  & $1.4 \times 10^{10}$ & 9 h 24 min \\
20   & $2 \times 10^{8}$  & $2.9 \times 10^{10}$ & 18 h 48 min \\
50   & $5 \times 10^{8}$  & $7.1 \times 10^{10}$ & \textbf{47 h (2 jours)} \\
100  & $1 \times 10^{9}$  & $1.4 \times 10^{11}$ & 94 h (4 jours) \\
\bottomrule
\end{tabular}
\end{table}

\subsection{Débit de dose}

Le débit de dose pour la source de \SI{42.4}{kBq} est :
\begin{equation}
\dot{D} = A \times D_{\text{par désint.}} = 4.24 \times 10^{4} \times 7 \times 10^{-2} = 2.97 \times 10^{3}~\text{nGy/s}
\end{equation}

Soit en unités plus pratiques :
\begin{equation}
\dot{D} = 2.97 \times 10^{3} \times 10^{-7} \times 3600 = \boxed{1.07~\text{cGy/h}}
\end{equation}

% ═══════════════════════════════════════════════════════════════════════════════
\section{Formules pratiques}
% ═══════════════════════════════════════════════════════════════════════════════

Pour une source d'Europium-152 de \SI{42.4}{kBq} et le dispositif puits couronne simulé :

\begin{center}
\fbox{
\parbox{0.8\textwidth}{
\centering
\textbf{Débit de dose :} $\dot{D} \approx \SI{1.07}{cGy/h}$

\vspace{0.3cm}

\textbf{Temps d'irradiation :} $t~(\text{h}) = \dfrac{D_{\text{cible}}~(\text{cGy})}{1.07}$

\vspace{0.3cm}

\textbf{Règle simplifiée :} $t \approx D_{\text{cible}}~(\text{cGy})$ en heures
}
}
\end{center}

\vspace{0.5cm}

\textbf{Exemples :}
\begin{itemize}
    \item Pour 5 cGy : $t \approx 5$ heures
    \item Pour 10 cGy : $t \approx 10$ heures
    \item Pour 50 cGy : $t \approx 50$ heures $\approx$ 2 jours
\end{itemize}

% ═══════════════════════════════════════════════════════════════════════════════
\section{Remarques}
% ═══════════════════════════════════════════════════════════════════════════════

\begin{enumerate}
    \item Ces calculs sont basés sur la dose dans l'anneau central (r = 0-5 mm). La dose est quasi-uniforme sur l'ensemble des anneaux (variation $<6\%$).
    
    \item L'activité de la source doit être corrigée si l'irradiation a lieu plusieurs mois/années après novembre 2025. Le facteur de correction est :
    \begin{equation}
        f = e^{-\frac{0.693}{13.517} \times \Delta t~(\text{ans})} \approx e^{-0.0513 \times \Delta t}
    \end{equation}
    
    \item Pour une activité différente $A'$, le temps d'irradiation se calcule par :
    \begin{equation}
        t' = t \times \frac{42.4~\text{kBq}}{A'}
    \end{equation}
\end{enumerate}

\end{document}
