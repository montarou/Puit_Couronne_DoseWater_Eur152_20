\documentclass[11pt,a4paper]{article}

% ═══════════════════════════════════════════════════════════════════════════════
% PACKAGES
% ═══════════════════════════════════════════════════════════════════════════════
\usepackage[utf8]{inputenc}
\usepackage[T1]{fontenc}
\usepackage[french]{babel}
\usepackage{amsmath,amssymb}
\usepackage{booktabs}
\usepackage{siunitx}
\usepackage{geometry}
\usepackage{xcolor}
\usepackage{tikz}
\usepackage{float}
\usepackage{caption}

\geometry{margin=2cm}
\sisetup{locale=FR}

\usetikzlibrary{patterns, arrows.meta, decorations.pathreplacing, calc, positioning}

% ═══════════════════════════════════════════════════════════════════════════════
% DOCUMENT
% ═══════════════════════════════════════════════════════════════════════════════

\begin{document}

\title{\textbf{Géométrie du dispositif Puits Couronne}\\[0.3cm]
\large Configuration avec PMMA 20 mm et feuille de tungstène}
\author{Simulation Geant4}
\date{\today}
\maketitle

% ═══════════════════════════════════════════════════════════════════════════════
\section{Schéma de la géométrie}
% ═══════════════════════════════════════════════════════════════════════════════

\begin{figure}[H]
\centering
\begin{tikzpicture}[scale=0.08, >=Stealth]

    % ═══════════════════════════════════════════════════════════════════════════
    % DÉFINITION DES COULEURS
    % ═══════════════════════════════════════════════════════════════════════════
    \definecolor{sourcecolor}{RGB}{255, 50, 50}
    \definecolor{filtercolor}{RGB}{120, 120, 140}
    \definecolor{containercolor}{RGB}{100, 100, 115}
    \definecolor{pmmacolor}{RGB}{255, 180, 100}
    \definecolor{watercolor}{RGB}{100, 150, 255}
    \definecolor{tungstencolor}{RGB}{60, 60, 60}
    \definecolor{precontainercolor}{RGB}{255, 140, 0}
    \definecolor{postcontainercolor}{RGB}{150, 0, 150}

    % ═══════════════════════════════════════════════════════════════════════════
    % AXE Z (vertical)
    % ═══════════════════════════════════════════════════════════════════════════
    \draw[->, thick] (-45, 0) -- (-45, 115) node[above] {$z$ (mm)};
    
    % Graduations
    \foreach \z in {0, 20, 40, 60, 80, 100} {
        \draw (-46, \z) -- (-44, \z);
        \node[left] at (-46, \z) {\footnotesize \z};
    }

    % ═══════════════════════════════════════════════════════════════════════════
    % SOURCE Eu-152 (z = 20 mm)
    % ═══════════════════════════════════════════════════════════════════════════
    \fill[sourcecolor] (0, 20) circle (3);
    \node[right] at (5, 20) {\footnotesize Source Eu-152};
    
    % Cône d'émission (20°)
    \draw[sourcecolor, dashed, thick] (0, 20) -- (-18, 70);
    \draw[sourcecolor, dashed, thick] (0, 20) -- (18, 70);
    \node[sourcecolor, right] at (12, 45) {\footnotesize $\theta = 20°$};

    % ═══════════════════════════════════════════════════════════════════════════
    % FILTRE W/PETG (z = 37.5 - 42.5 mm)
    % ═══════════════════════════════════════════════════════════════════════════
    \fill[filtercolor, opacity=0.8] (-25, 37.5) rectangle (25, 42.5);
    \draw[black, thick] (-25, 37.5) rectangle (25, 42.5);
    \node[right] at (28, 40) {\footnotesize Filtre W/PETG (5 mm)};
    
    % Cotation filtre
    \draw[<->, thin] (-30, 37.5) -- (-30, 42.5);
    \node[left] at (-30, 40) {\tiny 5};

    % ═══════════════════════════════════════════════════════════════════════════
    % PMMA (z = 76.5 - 96.5 mm) - MODIFIÉ : 20 mm
    % ═══════════════════════════════════════════════════════════════════════════
    \fill[pmmacolor, opacity=0.7] (-25, 76.5) rectangle (25, 96.5);
    \draw[black, thick] (-25, 76.5) rectangle (25, 96.5);
    \node[right] at (28, 86.5) {\footnotesize \textbf{PMMA (20 mm)}};
    
    % Cotation PMMA
    \draw[<->, thin] (-30, 76.5) -- (-30, 96.5);
    \node[left] at (-30, 86.5) {\tiny 20};

    % ═══════════════════════════════════════════════════════════════════════════
    % PreContainerPlane (z = 97 - 98 mm)
    % ═══════════════════════════════════════════════════════════════════════════
    \fill[precontainercolor, opacity=0.4] (-25, 97) rectangle (25, 98);
    \draw[precontainercolor, thick] (-25, 97) rectangle (25, 98);

    % ═══════════════════════════════════════════════════════════════════════════
    % EAU - 5 anneaux (z = 98.5 - 103.5 mm)
    % ═══════════════════════════════════════════════════════════════════════════
    % Anneau 0 (r = 0-5 mm)
    \fill[watercolor!90, opacity=0.8] (-5, 98.5) rectangle (5, 103.5);
    % Anneau 1 (r = 5-10 mm)
    \fill[watercolor!80, opacity=0.8] (-10, 98.5) rectangle (-5, 103.5);
    \fill[watercolor!80, opacity=0.8] (5, 98.5) rectangle (10, 103.5);
    % Anneau 2 (r = 10-15 mm)
    \fill[watercolor!70, opacity=0.8] (-15, 98.5) rectangle (-10, 103.5);
    \fill[watercolor!70, opacity=0.8] (10, 98.5) rectangle (15, 103.5);
    % Anneau 3 (r = 15-20 mm)
    \fill[watercolor!60, opacity=0.8] (-20, 98.5) rectangle (-15, 103.5);
    \fill[watercolor!60, opacity=0.8] (15, 98.5) rectangle (20, 103.5);
    % Anneau 4 (r = 20-25 mm)
    \fill[watercolor!50, opacity=0.8] (-25, 98.5) rectangle (-20, 103.5);
    \fill[watercolor!50, opacity=0.8] (20, 98.5) rectangle (25, 103.5);
    
    \draw[black, thick] (-25, 98.5) rectangle (25, 103.5);
    \node[right] at (28, 101) {\footnotesize Eau (5 mm, 5 anneaux)};
    
    % Cotation eau
    \draw[<->, thin] (-30, 98.5) -- (-30, 103.5);
    \node[left] at (-30, 101) {\tiny 5};

    % ═══════════════════════════════════════════════════════════════════════════
    % FEUILLE DE TUNGSTÈNE (z = 103.5 - 103.52 mm, 20 µm)
    % ═══════════════════════════════════════════════════════════════════════════
    \fill[tungstencolor] (-25, 103.5) rectangle (25, 104.5);  % Exagéré pour visibilité
    \draw[black] (-25, 103.5) rectangle (25, 104.5);
    \node[right] at (28, 104) {\footnotesize Feuille W (20 µm)};

    % ═══════════════════════════════════════════════════════════════════════════
    % PostContainerPlane (z = 102 - 103 mm, dans l'eau)
    % ═══════════════════════════════════════════════════════════════════════════
    \fill[postcontainercolor, opacity=0.4] (-25, 102) rectangle (25, 103);
    \draw[postcontainercolor, thick, dashed] (-25, 102) rectangle (25, 103);

    % ═══════════════════════════════════════════════════════════════════════════
    % CONTAINER W/PETG (parois latérales et couvercle)
    % ═══════════════════════════════════════════════════════════════════════════
    % Paroi gauche
    \fill[containercolor, opacity=0.8] (-27, 96.5) rectangle (-25, 103.5);
    % Paroi droite
    \fill[containercolor, opacity=0.8] (25, 96.5) rectangle (27, 103.5);
    % Couvercle
    \fill[containercolor, opacity=0.8] (-27, 103.5) rectangle (27, 105.5);
    
    \draw[black, thick] (-27, 96.5) -- (-27, 105.5) -- (27, 105.5) -- (27, 96.5);
    \draw[black, thick] (-25, 96.5) -- (-25, 103.5);
    \draw[black, thick] (25, 96.5) -- (25, 103.5);

    % ═══════════════════════════════════════════════════════════════════════════
    % LÉGENDE
    % ═══════════════════════════════════════════════════════════════════════════
    \node[anchor=north west] at (-45, -5) {
        \begin{tabular}{cl}
            \tikz\fill[sourcecolor] (0,0) circle (0.15); & Source Eu-152 \\
            \tikz\fill[filtercolor] (0,0) rectangle (0.4,0.2); & Filtre W/PETG \\
            \tikz\fill[pmmacolor] (0,0) rectangle (0.4,0.2); & PMMA \\
            \tikz\fill[watercolor] (0,0) rectangle (0.4,0.2); & Eau \\
            \tikz\fill[tungstencolor] (0,0) rectangle (0.4,0.2); & Feuille W \\
            \tikz\fill[containercolor] (0,0) rectangle (0.4,0.2); & Container W/PETG \\
        \end{tabular}
    };

    % ═══════════════════════════════════════════════════════════════════════════
    % ANNOTATIONS DES POSITIONS Z
    % ═══════════════════════════════════════════════════════════════════════════
    
    % Accolade pour le PMMA
    \draw[decorate, decoration={brace, amplitude=5pt, mirror}, thick] 
        (32, 76.5) -- (32, 96.5) node[midway, right=5pt] {\footnotesize 20 mm};
    
    % Flèche direction source
    \draw[->, thick, red] (0, 15) -- (0, 5);
    \node[red, below] at (0, 5) {\footnotesize vers source};
    
    % Flèche direction +z
    \draw[->, thick, blue] (0, 108) -- (0, 115);
    \node[blue, above] at (0, 115) {\footnotesize +z};

    % Distance PMMA - Filtre
    \draw[<->, dashed, gray] (0, 42.5) -- (0, 76.5);
    \node[gray, right] at (2, 59.5) {\footnotesize 34 mm (air)};

\end{tikzpicture}
\caption{Coupe schématique du dispositif Puits Couronne avec PMMA de 20 mm. L'épaisseur de la feuille de tungstène (20 µm) est exagérée pour la visualisation. La direction +z va de la source vers le détecteur.}
\label{fig:geometrie}
\end{figure}

% ═══════════════════════════════════════════════════════════════════════════════
\section{Tableau des positions des éléments du container}
% ═══════════════════════════════════════════════════════════════════════════════

\begin{table}[H]
\centering
\caption{Positions axiales (z) des éléments du dispositif avec PMMA de 20 mm}
\label{tab:positions}
\begin{tabular}{@{}lccccc@{}}
\toprule
\textbf{Élément} & \textbf{$z_{\min}$} & \textbf{$z_{\text{centre}}$} & \textbf{$z_{\max}$} & \textbf{Épaisseur} & \textbf{Matériau} \\
 & (mm) & (mm) & (mm) & (mm) & \\
\midrule
Source Eu-152 & -- & 20.0 & -- & ponctuelle & -- \\
\midrule
Filtre W/PETG & 37.5 & 40.0 & 42.5 & 5.0 & W/PETG 75\%/25\% \\
\midrule
Air (gap) & 42.5 & -- & 76.5 & 34.0 & Air \\
\midrule
\textbf{PMMA} & \textbf{76.5} & \textbf{86.5} & \textbf{96.5} & \textbf{20.0} & PMMA \\
\midrule
PreContainerPlane & 97.0 & 97.5 & 98.0 & 1.0 & Air \\
\midrule
Eau (anneaux) & 98.5 & 101.0 & 103.5 & 5.0 & H$_2$O \\
\midrule
PostContainerPlane & 102.0 & 102.5 & 103.0 & 1.0 & H$_2$O \\
\midrule
Feuille tungstène & 103.500 & 103.510 & 103.520 & 0.020 & W pur \\
\midrule
Container (couvercle) & 103.5 & 104.5 & 105.5 & 2.0 & W/PETG \\
\bottomrule
\end{tabular}
\end{table}

% ═══════════════════════════════════════════════════════════════════════════════
\section{Paramètres géométriques}
% ═══════════════════════════════════════════════════════════════════════════════

\begin{table}[H]
\centering
\caption{Paramètres géométriques du dispositif}
\label{tab:parametres}
\begin{tabular}{@{}lrl@{}}
\toprule
\textbf{Paramètre} & \textbf{Valeur} & \textbf{Unité} \\
\midrule
Position centre container ($z_{\text{container}}$) & 100.0 & mm \\
Hauteur intérieure container & 7.0 & mm \\
Épaisseur parois container & 2.0 & mm \\
\midrule
Épaisseur eau & 5.0 & mm \\
\textbf{Épaisseur PMMA} & \textbf{20.0} & \textbf{mm} \\
Épaisseur feuille W & 20 & µm \\
Épaisseur plans de comptage & 1.0 & mm \\
\midrule
Rayon intérieur container & 25.0 & mm \\
Rayon PMMA & 25.0 & mm \\
Rayon feuille W & 25.0 & mm \\
\bottomrule
\end{tabular}
\end{table}

% ═══════════════════════════════════════════════════════════════════════════════
\section{Calcul des positions avec PMMA de 20 mm}
% ═══════════════════════════════════════════════════════════════════════════════

Les positions sont calculées de haut en bas dans le container :

\begin{align}
z_{\text{eau, haut}} &= z_{\text{container}} + \frac{h_{\text{container}}}{2} = 100 + 3.5 = 103.5~\text{mm} \\
z_{\text{eau, centre}} &= z_{\text{eau, haut}} - \frac{e_{\text{eau}}}{2} = 103.5 - 2.5 = 101.0~\text{mm} \\
z_{\text{eau, bas}} &= z_{\text{eau, haut}} - e_{\text{eau}} = 103.5 - 5 = 98.5~\text{mm}
\end{align}

Position du PMMA (extension vers la source avec épaisseur 20 mm) :
\begin{align}
z_{\text{PMMA, haut}} &= z_{\text{eau, bas}} - e_{\text{PreContainer}} = 98.5 - 1.0 - 1.0 = 96.5~\text{mm} \\
z_{\text{PMMA, centre}} &= z_{\text{PMMA, haut}} - \frac{e_{\text{PMMA}}}{2} = 96.5 - 10.0 = \boxed{86.5~\text{mm}} \\
z_{\text{PMMA, bas}} &= z_{\text{PMMA, haut}} - e_{\text{PMMA}} = 96.5 - 20.0 = \boxed{76.5~\text{mm}}
\end{align}

Position de la feuille de tungstène :
\begin{align}
z_{\text{W, bas}} &= z_{\text{eau, haut}} = 103.5~\text{mm} \\
z_{\text{W, haut}} &= z_{\text{W, bas}} + e_{\text{W}} = 103.5 + 0.02 = 103.52~\text{mm}
\end{align}

% ═══════════════════════════════════════════════════════════════════════════════
\section{Comparaison des configurations}
% ═══════════════════════════════════════════════════════════════════════════════

\begin{table}[H]
\centering
\caption{Comparaison des configurations PMMA 5 mm, 10 mm et 20 mm}
\label{tab:comparaison}
\begin{tabular}{@{}lcccc@{}}
\toprule
\textbf{Paramètre} & \textbf{PMMA 5 mm} & \textbf{PMMA 10 mm} & \textbf{PMMA 20 mm} & \textbf{$\Delta$ (5$\to$20)} \\
\midrule
Épaisseur PMMA & 5.0 mm & 10.0 mm & 20.0 mm & +15.0 mm \\
$z_{\text{PMMA, bas}}$ & 91.5 mm & 86.5 mm & 76.5 mm & -15.0 mm \\
$z_{\text{PMMA, haut}}$ & 96.5 mm & 96.5 mm & 96.5 mm & 0 mm \\
$z_{\text{PMMA, centre}}$ & 94.0 mm & 91.5 mm & 86.5 mm & -7.5 mm \\
\midrule
Distance source-PMMA & 71.5 mm & 66.5 mm & 56.5 mm & -15.0 mm \\
Gap air (filtre-PMMA) & 49.0 mm & 44.0 mm & 34.0 mm & -15.0 mm \\
\bottomrule
\end{tabular}
\end{table}

\textbf{Note :} L'augmentation de l'épaisseur du PMMA se fait vers la source (côté $-z$). La face supérieure du PMMA reste fixe à $z = 96.5$ mm, tandis que la face inférieure descend de 91.5 mm (5 mm) à 76.5 mm (20 mm).

% ═══════════════════════════════════════════════════════════════════════════════
\section{Empilement des éléments (direction +z)}
% ═══════════════════════════════════════════════════════════════════════════════

L'ordre des éléments depuis la source vers le détecteur est :

\begin{enumerate}
    \item \textbf{Source Eu-152} : $z = 20$ mm
    \item \textbf{Filtre W/PETG} : $z = 37.5 - 42.5$ mm (5 mm)
    \item \textbf{Air} : $z = 42.5 - 76.5$ mm (34 mm)
    \item \textbf{PMMA} : $z = 76.5 - 96.5$ mm (\textbf{20 mm})
    \item \textbf{PreContainerPlane} : $z = 97 - 98$ mm (1 mm, air)
    \item \textbf{Eau} : $z = 98.5 - 103.5$ mm (5 mm, 5 anneaux)
    \item \textbf{Feuille W} : $z = 103.5 - 103.52$ mm (20 µm)
    \item \textbf{Couvercle container} : $z = 103.5 - 105.5$ mm (2 mm)
\end{enumerate}

% ═══════════════════════════════════════════════════════════════════════════════
\section{Modification du code DetectorConstruction.cc}
% ═══════════════════════════════════════════════════════════════════════════════

La modification à effectuer dans le fichier \texttt{DetectorConstruction.cc} est à la ligne 31 :

\begin{verbatim}
// AVANT (configuration originale)
fPMMAThickness(5.0*mm),           // épaisseur PMMA

// APRÈS (nouvelle configuration 20 mm)
fPMMAThickness(20.0*mm),          // MODIFIÉ : épaisseur PMMA (5 mm -> 20 mm)
\end{verbatim}

Le reste du code calcule automatiquement les positions en fonction de cette épaisseur :
\begin{verbatim}
G4double pmmaTopZ = waterBottomZ - fCountingPlaneThickness;  // Fixé à 96.5 mm
G4double pmmaCenterZ = pmmaTopZ - fPMMAThickness/2;          // 86.5 mm
G4double pmmaBottomZ = pmmaTopZ - fPMMAThickness;            // 76.5 mm
\end{verbatim}

\end{document}
